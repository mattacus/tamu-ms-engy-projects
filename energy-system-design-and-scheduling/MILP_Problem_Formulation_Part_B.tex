\documentclass{article}
\usepackage{amsmath}
\usepackage{amssymb}
\usepackage[margin=1in]{geometry}
\usepackage{booktabs} % For better table lines
\usepackage{caption}  % For table captions
\usepackage{float}    % Add this package for [H] placement
\usepackage{siunitx}  % For formatting units and numbers
\usepackage{enumitem} % For custom list formatting

\begin{document}

\title{ENGY 604 Project - Multi-Period MILP Formulation (Part B - Scheduling)}
\author{Team 3}
\date{}
\maketitle

\section*{1. Sets}
\begin{itemize}
    \item $T$: Set of Time Periods, $t \in \{1, 2, \ldots, 24\}$ (hourly periods in a day)
    \item $R$: Set of Primary Resources, $r \in \{\text{Natural Gas, Biomass, Grid Electricity}\}$
    \item $G$: Set of Generators, $g \in \{\text{Biomass ST, NatGas CHP, Solar PV, Wind Farm}\}$
    \item $D$: Set of Conversion Devices, $d \in \{\text{Refrigerator, LED, Heater}\}$
\end{itemize}

\section*{2. Parameters (Base)}

\subsection*{General Parameters}
\begin{itemize}
    \item $H_{\text{day}}$: Operating horizon (1 day = 24 hours)
    \item $\Delta t$: Duration of each time period (1 hour)
\end{itemize}

\subsection*{Resource Parameters}
\begin{itemize}
    \item $\text{Price}_r$: Price of resource $r$ (\$/GJ) - converted to \$/kWh in Appendix B
\end{itemize}

\subsection*{Generator Parameters}
\begin{itemize}
    \item $\text{capex}_g$: Capital cost for generator $g$ (\$/kW) - \textit{assumed to be sunk cost}
    \item $\text{opex}_g$: Operational cost for generator $g$ (\$/kW-year) - converted to \$/kWh in Appendix B
    \item $\eta_g^{\text{elec}}$: Electrical efficiency of generator $g$ (as fraction)
    \item $\eta_g^{\text{heat}}$: Heating efficiency of generator $g$ (as fraction)
    \item $C_g^{\text{min}}$: Minimum capacity of generator $g$ (kW)
    \item $C_g^{\text{max}}$: Maximum capacity of generator $g$ (kW)
    \item $C_g^{\text{installed}}$: Installed (fixed) capacity of generator $g$ (kW) - \textit{pre-determined}
\end{itemize}

\subsection*{Conversion Devices Parameters}
\begin{itemize}
    \item $\eta_d$: Efficiency of conversion device $d$ (as fraction, or COP)
    \item $\text{Demand}_{d,t}$: Demand for conversion device $d$ at time $t$ (kW)
          \begin{itemize}
              \item $\text{Demand}_{\text{refrig},t} = 1000$ kW (constant for all $t$)
              \item $\text{Demand}_{\text{led},t}$: Time-varying (see Table~\ref{tab:lighting_demand})
              \item $\text{Demand}_{\text{heater},t} = 100$ kW (constant for all $t$)
          \end{itemize}
\end{itemize}

\section*{3. Parameters (Intermediate)}

\subsection*{Total Electrical Demand at Time $t$ (kW)}
This is the total power required by the electricity-driven conversion devices at each time period.
\begin{align*}
    \text{ElecDemand}_t & = \frac{\text{Demand}_{\text{refrig},t}}{\eta_{\text{refrig}}} + \frac{\text{Demand}_{\text{led},t}}{\eta_{\text{led}}} \quad \forall t \in T
\end{align*}

\subsection*{Total Heat Demand at Time $t$ (kW)}
This is the total heat required by the heat-driven conversion devices at each time period.
\begin{align*}
    \text{HeatDemand}_t & = \frac{\text{Demand}_{\text{heater},t}}{\eta_{\text{heater}}} \quad \forall t \in T
\end{align*}

\newpage

\section*{4. Mathematical Problem Formulation}

\subsection*{4.1 Decision Variables}
\begin{description}[style=multiline, leftmargin=3cm, font=\normalfont]
    \item[$P_{g,t}^{\text{elec}} $]: Continuous variable for the actual electric power produced by generator $g$ at time $t$ (kW).
    \item[$P_{r,t}^{\text{cons}} $]: Continuous variable for the amount of primary resource $r$ consumed at time $t$, in equivalent units of power (kW).
\end{description}

\subsection*{4.2 Objective Function}
\begin{description}[style=multiline, leftmargin=3cm, font=\normalfont]
    \item[Objective:] Minimize Total Daily Operating Cost ($Z_{\text{COST}}$)
          \begin{equation*}
              \min Z_{\text{COST}} = \underbrace{\sum_{t \in T} \sum_{g \in G} \left(P_{g,t}^{\text{elec}} \cdot \Delta t \cdot \text{opex}_g\right)}_{\text{Generator Operating Cost}} + \underbrace{\sum_{t \in T} \sum_{r \in R} \left(P_{r,t}^{\text{cons}} \cdot \Delta t \cdot \text{Price}_r\right)}_{\text{Resource (Fuel) Cost}}
          \end{equation*}

          \noindent where $\Delta t = 1$ hour for each time period.
\end{description}

\subsection*{4.3 Constraints}
\smallskip
\noindent \textbf{Note:} The installed capacity $C_g^{\text{installed}}$ for each generator is assumed to be pre-determined (a fixed parameter). Wind and solar have no resource costs.

\begin{description}[style=multiline, leftmargin=3cm, font=\normalfont]
    \item[C1:] Electricity Balance (for each time period)
          \begin{equation*}
              \sum_{g \in G} P_{g,t}^{\text{elec}} + P_{\text{Grid},t}^{\text{cons}} = \text{ElecDemand}_t \quad \forall t \in T
          \end{equation*}

    \item[C2:] Heat Balance (for each time period)
          \begin{equation*}
              P_{\text{CHP},t}^{\text{elec}} \cdot \frac{\eta_{\text{CHP}}^{\text{heat}}}{\eta_{\text{CHP}}^{\text{elec}}} = \text{HeatDemand}_t \quad \forall t \in T
          \end{equation*}
          \noindent where $P_{\text{CHP},t}^{\text{heat}} = P_{\text{CHP},t}^{\text{elec}} \cdot \frac{\eta_{\text{CHP}}^{\text{heat}}}{\eta_{\text{CHP}}^{\text{elec}}}$

    \item[C3:] Fuel Balance - Biomass (for each time period)
          \begin{equation*}
              P_{\text{BioST},t}^{\text{elec}} = P_{\text{Biomass},t}^{\text{cons}} \cdot \eta_{\text{BioST}}^{\text{elec}} \quad \forall t \in T
          \end{equation*}

    \item[C4:] Fuel Balance - Natural Gas (for each time period)
          \begin{equation*}
              P_{\text{CHP},t}^{\text{elec}} = P_{\text{NatGas},t}^{\text{cons}} \cdot \eta_{\text{CHP}}^{\text{elec}} \quad \forall t \in T
          \end{equation*}

    \item[C5:] Generation Output Limit (for each time period)
          \begin{equation*}
              P_{g,t}^{\text{elec}} \le C_g^{\text{installed}} \quad \forall g \in G, \forall t \in T
          \end{equation*}

    \item[C6:] Non-Negativity
          \begin{align*}
              P_{g,t}^{\text{elec}}, P_{r,t}^{\text{cons}} & \ge 0 \quad \forall g \in G, \forall r \in R, \forall t \in T
          \end{align*}
\end{description}

\newpage
\appendix
\section*{Appendix A: Parameter Value Tables}

\begin{table}[H]
    \centering
    \caption{Time-varying lighting demand for each hour of the day}
    \label{tab:lighting_demand}
    \begin{tabular}{@{}cc|cc@{}}
        \toprule
        Time Period & Lighting Demand [kW] & Time Period & Lighting Demand [kW] \\ \midrule
        1           & 91                   & 13          & 159                  \\
        2           & 88                   & 14          & 163                  \\
        3           & 87                   & 15          & 165                  \\
        4           & 88                   & 16          & 166                  \\
        5           & 92                   & 17          & 168                  \\
        6           & 101                  & 18          & 176                  \\
        7           & 112                  & 19          & 183                  \\
        8           & 123                  & 20          & 180                  \\
        9           & 132                  & 21          & 172                  \\
        10          & 141                  & 22          & 165                  \\
        11          & 150                  & 23          & 156                  \\
        12          & 155                  & 24          & 147                  \\ \bottomrule
    \end{tabular}
\end{table}

\begin{table}[H]
    \centering
    \caption{Technical and economic parameters of on-site energy conversion technologies}
    \label{tab:table1}
    \begin{tabular}{@{}lccc@{}}
        \toprule
        Sources         & Refrigerator  & LED         & Heater        \\ \midrule
        Input Resource  & Electricity   & Electricity & Heat          \\
        Output Resource & Refrigeration & Lighting    & Space Heating \\
        $\eta_d$ [\%]   & 300 (COP)     & 80          & 85            \\ \bottomrule
    \end{tabular}
\end{table}

\begin{table}[H]
    \centering
    \caption{Technical and economic parameters of energy generation technologies}
    \label{tab:table2}
    \begin{tabular}{@{}lcccc@{}}
        \toprule
        Processes                     & BioST             & CHP                 & PV          & Wind        \\ \midrule
        Input Resource                & Biomass           & Natural Gas         & Solar       & Wind        \\
        Output Resource               & Electricity       & Electricity \& Heat & Electricity & Electricity \\
        $\eta_g^{\text{elec}}$ [\%]   & 68                & 44                  & 9           & 22          \\
        $\eta_g^{\text{heat}}$ [\%]   & 0                 & 28                  & 0           & 0           \\
        $C_g^{\text{min}}$ [kW]       & 100               & 800                 & 10          & 10          \\
        $C_g^{\text{max}}$ [kW]       & $1 \times 10^{6}$ & $1 \times 10^{6}$   & 300         & 500         \\
        $C_g^{\text{installed}}$ [kW] & TBD               & TBD                 & TBD         & TBD         \\
        $\text{capex}_g$ [\$/kW]      & 250               & 500                 & 2000        & 2000        \\
        $\text{opex}_g$ [\$/kW-year]  & 15                & 15                  & 500         & 1200        \\ \bottomrule
    \end{tabular}
\end{table}

\begin{table}[H]
    \centering
    \caption{Prices of the primary energy resources and grid electricity}
    \label{tab:table3}
    \begin{tabular}{@{}lccc@{}}
        \toprule
        Sources                  & Natural Gas & Biomass & Grid Electricity \\ \midrule
        $\text{Price}_r$ [\$/GJ] & 8.89        & 9.72    & 36.11            \\ \bottomrule
    \end{tabular}
\end{table}

\begin{table}[H]
    \centering
    \caption{Energy demands for various utilities (constant demands)}
    \label{tab:table4}
    \begin{tabular}{@{}lcc@{}}
        \toprule
        Uses                   & Refrigeration & Space Heating \\ \midrule
        $\text{Demand}_d$ [kW] & 1000          & 100           \\ \bottomrule
    \end{tabular}
\end{table}

\newpage
\section*{Appendix B: Sample Intermediate Parameter Calculations}

\noindent These are sample calculations for selected time periods to illustrate the intermediate parameters.

\subsection*{Resource Price Conversions (\$/GJ to \$/kWh)}
Converting from \$/GJ to \$/kWh using the conversion factor $1 \text{ GJ} = 277.78 \text{ kWh}$ (or equivalently, divide by 3600 since 1 kWh = 3.6 MJ):
\begin{align*}
    \text{Price}_{\text{natgas}}  & = \frac{8.89 \text{ \$/GJ}}{277.78 \text{ kWh/GJ}} = 0.00247 \text{ \$/kWh}  \\
    \text{Price}_{\text{biomass}} & = \frac{9.72 \text{ \$/GJ}}{277.78 \text{ kWh/GJ}} = 0.00270 \text{ \$/kWh}  \\
    \text{Price}_{\text{grid}}    & = \frac{36.11 \text{ \$/GJ}}{277.78 \text{ kWh/GJ}} = 0.01003 \text{ \$/kWh}
\end{align*}

\subsection*{Generator OPEX Conversions (\$/kW-year to \$/kWh)}
Converting from \$/kW-year to \$/kWh by dividing by 8760 hours/year:
\begin{align*}
    \text{opex}_{\text{BioST}} & = \frac{15 \text{ \$/kW-year}}{8760 \text{ hours/year}} = 0.00171 \text{ \$/kWh}   \\
    \text{opex}_{\text{CHP}}   & = \frac{15 \text{ \$/kW-year}}{8760 \text{ hours/year}} = 0.00171 \text{ \$/kWh}   \\
    \text{opex}_{\text{PV}}    & = \frac{500 \text{ \$/kW-year}}{8760 \text{ hours/year}} = 0.05707 \text{ \$/kWh}  \\
    \text{opex}_{\text{Wind}}  & = \frac{1200 \text{ \$/kW-year}}{8760 \text{ hours/year}} = 0.13699 \text{ \$/kWh}
\end{align*}

\subsection*{Total Electrical Demand (kW) - Sample Calculations}
For $t=1$ (Time Period 1):
\begin{align*}
    \text{ElecDemand}_1 & = \frac{\text{Demand}_{\text{refrig},1}}{\eta_{\text{refrig}}} + \frac{\text{Demand}_{\text{led},1}}{\eta_{\text{led}}} \\
                        & = \frac{1000 \text{ kW}}{3.0} + \frac{91 \text{ kW}}{0.80}                                                              \\
                        & = 333.33 \text{ kW} + 113.75 \text{ kW}                                                                                 \\
                        & = 447.08 \text{ kW}
\end{align*}

For $t=19$ (Time Period 19 - Peak Lighting):
\begin{align*}
    \text{ElecDemand}_{19} & = \frac{1000 \text{ kW}}{3.0} + \frac{183 \text{ kW}}{0.80} \\
                           & = 333.33 \text{ kW} + 228.75 \text{ kW}                     \\
                           & = 562.08 \text{ kW}
\end{align*}

\subsection*{Total Heat Demand (kW)}
This remains constant for all time periods:
\begin{align*}
    \text{HeatDemand}_t & = \frac{\text{Demand}_{\text{heater},t}}{\eta_{\text{heater}}} \\
                        & = \frac{100 \text{ kW}}{0.85}                                  \\
                        & = 117.65 \text{ kW} \quad \forall t \in T
\end{align*}

\subsection*{Key Notes for Multi-Period Model}
\begin{itemize}
    \item The installed capacity $C_g^{\text{installed}}$ must be specified before solving the scheduling problem. This represents the generation capacity already built and available for dispatch.
    \item Capital costs (CAPEX) are not included in the objective function as they are considered sunk costs.
    \item The model determines the optimal hourly dispatch of generation resources to meet time-varying demands at minimum operating cost.
    \item Lighting demand varies from 87 kW (minimum at $t=3$) to 183 kW (maximum at $t=19$).
    \item Total electrical demand varies from approximately 442 kW to 562 kW across the 24-hour period.
\end{itemize}

\end{document}
